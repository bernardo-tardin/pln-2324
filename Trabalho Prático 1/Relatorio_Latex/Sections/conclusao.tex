\section{Conclusão}
Para a realização deste trabalho foi essencial estabelecer, primeiramente, uma análise cuidada da estrutura de todos os documentos, de maneira a tornar possível a identificação de padrões que permitem distinguir determinadas secções ou componentes.\\
Após esta fase de identificação, foi possível dar início à fase de limpeza, onde foi colocado em prática todo o conhecimento adquirido ao longo das aulas no que se refere à escrita de expressões regulares.\\

O trabalho em causa permitiu, efetivamente, aprimorar a capacidade de escrita destas expressões, uma vez que foram encontrados os mais diversos padrões, cada um com o seu grau de distinção.\\
Além do mais, tornou-se evidente todo o poder e versatilidade das expressões regulares, uma vez que, com uma sintaxe adequada, e uso de caracteres, e metacaracteres especiais, é possível criar padrões de alta complexidade e efetuar operações em texto com um elevado grau de eficiência.\\

No entanto, uma das dificuldades encontradas consistiu no estabelecimento da generalidade das expressões regulares de forma a extrair apenas a informação necessária, sem envolver a escrita de expressões em demasia. \\

Assim sendo, as expressões regulares desenvolvidas ao longo do projeto foram estabelecidas de maneira a serem de fácil compreensão e implementação, evitando a possível influência sobre outros elementos. Com isto, procedeu-se à construção dos diferentes ficheiros JSON, podendo-se concluir, assim, que o objetivo do trabalho foi alcançado.

