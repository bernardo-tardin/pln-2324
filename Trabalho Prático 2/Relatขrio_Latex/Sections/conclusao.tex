\section{Conclusão}
Ao longo deste trabalho, foi possível explorar detalhadamente o processo de desenvolvimento de uma plataforma \textit{web}, através do uso de duas ferramentas cruciais: \textit{Flask} e \textit{Jinja}. No que se refere à primeira, esta permitiu a construção de uma aplicação robusta, enquanto que a segunda facilitou, consideravelmente, a criação das várias interfaces dinâmicas e interativas.\\

Inicialmente, revelou-se essencial a obtenção de informações adicionais, através de \textit{Web Scraping}, relativamente aos conjuntos de dados utilizados no trabalho prático anterior. De facto, esta técnica permitiu enriquecer a aplicação com dados atualizados e relevantes, melhorando, significativamente, a experiência do utilizador ao fornecer informações mais precisas e contextualizadas.\\

De seguida, procedeu-se, então, ao desenvolvimento dos demais \textit{templates}, tendo sido este processo agilizado graças à \textit{framework Bootstrap}. Todavia, durante este procedimento, revelou-se essencial a implementação de testes, e metodologias ágeis, de maneira a garantir a qualidade da plataforma. \\

Com isto, ao longo do presente relatório, foram demonstradas, e esclarecidas, todas as funcionalidades que a aplicação engloba: desde uma filtragem dos demais conceitos por categorias, a adição, ou eliminação, de um determinado termo, até à alteração dos parâmetros descritivos de um conceito, como a sua descrição e tradução.\\

Em suma, a elaboração deste projeto revelou-se uma experiência enriquecedora, uma vez que proporcionou uma compreensão aprofundada das possibilidades oferecidas por todas as ferramentas utilizadas. Através deste trabalho, foi possível aplicar todo o conhecimento adquirido ao longo da Unidade Curricular, o que facilitou a sua execução. Assim, o projeto não apresentou um elevado grau de dificuldade, uma vez que envolve tópicos já abordados, e praticados, durante as aulas.

\section{Bibliografia}
\printbibliography




